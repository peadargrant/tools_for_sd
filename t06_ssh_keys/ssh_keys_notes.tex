\documentclass[slides]{pgnotes}

\title{SSH keys}

\begin{document}

\maketitle

\section{Key-based authentication}
\label{key-based-authentication}

SSH key pairs are an alternative to a username/password.
They consist of:

\begin{description}
\item[Private key]
kept on the client and securely stored.
\item[Public key]
on the server(s) you want to log in to. (The public key can be freely
shared around, even put up in public.)
\end{description}

\autoimage{key_pair}{Key-based authentication}{ssh-key-authentication}


\subsection{Creating key pair (mac, linux, windows 10 onwards)}
\label{sec:creating-key-pair-unix}

Key pairs are normally and ideally created on your own local client computer.
Key pairs only need to be generated once.
(If you already have a key pair created, you can skip on ahead...)

To create an ED25519 key pair, in Powershell/Bash type:

\begin{verbatim}
ssh-keygen -t ed25519
\end{verbatim}

You can optionally use a passphrase to encrypt the key pair or leave it blank for easier usage.

\subsection{Default key locations}

The key pair is then stored in two files in your home directory (same for Mac, Linux, Windows).
You can find them by changing into the \texttt{.ssh} directory and listing the contents of it:

\begin{verbatim}
cd .ssh
dir
\end{verbatim}

From the directory listing:

\begin{verbatim}
Mode                LastWriteTime         Length Name
----                -------------         ------ ----
-a----       16/10/2020     15:19           3243 id_ed25519
-a----       16/10/2020     15:19            749 id_ed25519.pub
\end{verbatim}

The public key is stored in \texttt{id\_ed25519.pub}.
The private key is stored in \texttt{id\_ed25519}.

You can of course copy these files to/from a memory stick or online storage.
Remember though that if your private key is compromised, anybody can use it.
Best to protect it with a passphrase!

% \subsection{Creating key pair (windows using PuTTYGen)}
% \label{sec:creating-key-pair-windows}

% PuTTYgen is a key generator tool for Windows.
% It's bundled with PuTTY.
% Start PuTTYgen.
% Chose Ed25519 as the key type and hit Generate.
% Move the mouse over the blank area to create some randomness.
% Then save the Private Key file to a known location on your computer.


\subsection{Connecting over SSH with keys}
\label{sec:connecting-over-ssh-with-keys}

In PowerShell/Bash we can use the SSH command to connect to the SSH server on a remote host:
\begin{itemize}
\item This will then present us with a new shell on the remote computer (Bash for Linux/UNIX, PowerShell for Windows).
\item By default, SSH will try all private keys in \texttt{.ssh} so we don't need to specify which.
\end{itemize}

\begin{verbatim}
ssh student@$publicIp 
\end{verbatim}


\subsection{Using specific key}

We can force the use of a specific key using the \texttt{-i} option:

\begin{minted}{bash}
ssh -i private_key_file_name_here username@host
\end{minted}

\end{document}