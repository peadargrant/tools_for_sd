\documentclass[slides]{pgnotes}

\title{SSH keys}

\begin{document}

\maketitle

\section{SSH authentication}

Available methods:
\begin{description}
\item[Password] typed on demand (as you've used so far).
\item[Key pair] using cryptographic key files (today's focus!)
  \begin{itemize}
  \item Some hardware crypto devices use a hardware private key.
  \end{itemize}
\item[2-Factor Authentication] using authenticator app.
\item[One-Time Password] from a printed list.
\item[Kerberos / GSSAPI] to pass through credentials (advanced!)
\end{description}

Which methods are \textit{required} and \textit{sufficient} will depend on how the SSH server and the operating system on the remote host are configured!

\section{Key-based authentication}
\label{key-based-authentication}

SSH key pairs are an alternative or addition to a username/password.
They consist of:

\begin{description}
\item[Private key]
kept on the client and securely stored.
\item[Public key]
on the server(s) you want to log in to. (The public key can be freely
shared around, even put up in public.)
\end{description}

\autoimage{key_pair}{Key-based authentication}{ssh-key-authentication}

\subsection{Motivation}

Main reasons to use a public / private key pair:

\begin{itemize}
\item Avoids repetitive typing of password when logging into a host.
\item Provides security against \textit{brute force} password attacks.
\end{itemize}

There are also some reported security downsides to SSH Key Pairs!

\begin{itemize}
\item No universal means to revoke keys if a private key is stolen.
\end{itemize}


\subsection{Encryption algorithms}

There are a number of encryption algorithms:
\begin{description}
\item[RSA:] at least 4096-bits
\item[ED25519:] highly recommended!
\item[DSA:] not recommended anymore
\item[ECDSA:] not recommended anymore
\end{description}
You don't need to worry too much about their mechanics.
I generally choose \texttt{ed25519}.

\subsection{Creating key pair (mac, linux, windows 10 onwards)}
\label{sec:creating-key-pair-unix}

Key pairs are normally and ideally created on your own local client computer.
Key pairs only need to be generated once.

To create an ED25519 key pair, in Powershell/Bash type:

\begin{verbatim}
ssh-keygen -t ed25519
\end{verbatim}

Just press enter for now at each prompt.


\subsection{Default key locations}

The \textbf{key pair} is stored in \textbf{two} files (same for Mac, Linux, Windows).
You can find them by changing into the \texttt{.ssh} direcxstory and listing the contents of it:

\begin{verbatim}
cd .ssh
ls
\end{verbatim}

From the directory listing:

\begin{verbatim}
Mode                LastWriteTime         Length Name
----                -------------         ------ ----
-a----       16/10/2020     15:19           3243 id_ed25519 <-- PRIVATE
-a----       16/10/2020     15:19            749 id_ed25519.pub <-- public
\end{verbatim}


You can of course copy these files to/from a memory stick or online storage.
Remember though that if your private key is compromised, anybody can use it.


% \subsection{Creating key pair (windows using PuTTYGen)}
% \label{sec:creating-key-pair-windows}

% PuTTYgen is a key generator tool for Windows.
% It's bundled with PuTTY.
% Start PuTTYgen.
% Chose Ed25519 as the key type and hit Generate.
% Move the mouse over the blank area to create some randomness.
% Then save the Private Key file to a known location on your computer.


\subsection{Connecting over SSH with keys}
\label{sec:connecting-over-ssh-with-keys}

In PowerShell/Bash we can use the SSH command to connect to the SSH server on a remote host:
\begin{itemize}
\item This will then present us with a new shell on the remote computer (Bash for Linux/UNIX, PowerShell for Windows).
\item By default, SSH will try all private keys in \texttt{.ssh} so we don't need to specify which.
\end{itemize}

\begin{verbatim}
ssh student@$publicIp 
\end{verbatim}

All you will notice is that you're not asked for a password. 

\subsection{Using specific non-default key}

We can force the use of a specific key using the \texttt{-i} option:

\begin{minted}{bash}
ssh -i private_key_file_name_here username@host
\end{minted}

This is useful if a key is kept on a USB stick or network device.


\section{Encrypted private keys}

You can optionally encrypt the private key file with a passphrase (aka password) when creating it.

The passphrase will be needed any time the private key file is used.


\subsection{SSH agent}

The SSH agent is a program that allows a private key to be decrypted once and used a number of times while logged in.


\section{Private key security}

\begin{center}
  \textbf{\Large Your private key is valuable!}
\end{center}

\textbf{Important points:}

\begin{itemize}
\item If an unauthorized person has your private key, they can log into any server that your public key is on.
\item Best to protect your private key using a passphrase (where possible!)
  \begin{itemize}
  \item Alternatives like private key on full disk encrypted laptop ok too!
  \end{itemize}
\end{itemize}

\subsection{Risk factors}

Your private key is more at risk of compromise when the following risk factors apply:

\begin{itemize}
\item The private key file has \textbf{no passphrase} to protect it if it falls into the wrong hands.
\item There are copies of your private key in a large number of locations.
\item You use your private key (and passphrase) on client systems that you don't trust.
\end{itemize}

\end{document}