\documentclass[slides]{pgnotes}

\title{Scripting variables}

\begin{document}

\maketitle

\tableofcontents

\section{Shell Variables}\label{shell-variables}

\begin{itemize}
  
\item
  Variables are used to store values in a script.
\item
  You can use shell variables when in Bash interactively too!
\item
  Can be used more than once and can be reassigned a value.
\item
  BASH has two main types of Variables:
  \begin{description}
  \item[Environment variables] that contain information that the system and programs access regularly
  \item[User-defined variables] defined by users 
  \end{description}
\end{itemize}


\section{Environment Variables}\label{environment-variables}

\begin{itemize}

\item
  Examples of Environment Variables include \texttt{\$HOME, \$HOSTNAME, \$PWD, \$SHELL}
\item
  To view the available Environment Variables, use the \texttt{printenv} command
\item
  To view environment variable value, use \texttt{echo} (which outputs to screen):\\
  \texttt{echo \$HOSTNAME}
\item
  You can also use the \texttt{set} and \texttt{env} commands to view the environment variables and their values
\end{itemize}

\section{PATH Environment Variable}\label{path-environment-variable}

\begin{itemize}

\item
  The PATH variable is one of the most important variables in the BASH shell
\item
  Allows users to execute commands by typing the command name alone
\item
  If a command is located within a directory that is listed in the PATH variable,
  You can type the name of the command on the command line to execute it
\item
  To view contents of PATH variable:
  \begin{itemize}
  \item
    Interactively: \texttt{\$PATH }
  \item
    Or in script \texttt{echo \$PATH}
  \end{itemize}
\end{itemize}

\section{User Defined Variables}\label{user-defined-variables}

\begin{itemize}
\item
  Variable Names are important and should reflect the name of the value where possible.
\item
  No need to declare a variable, just assign a value to it (same as Python)
\item
  To assign a value to a variable use the =equal sign:\\
  \mintinline{bash}{college=dkit}
\item
  Here we have created a \textbf{variable} called \textbf{college} and assigned (\textbf{=)}) the value \textbf{dkit} to it
\item
  To use a variable value, use the \$ sign in front of the variable:\\
  \mintinline{bash}{echo ``The college name is'' \$college}
\end{itemize}

\subsection{Variable names}
\begin{itemize}
\item
  Variables can contain alphanumeric characters, the dash character, or the underscore character.
\item
  They must not start with a number.
\item
  Better not to use UPPERCASE.
\item
  Decide on appropriate naming convention i.e. surName or firstName or fname and stick with this convention.
\item To display the values of the  variables, put the \textbf{\$} sign infront of the variable.
\end{itemize}

\inputminted{bash}{usage.sh}


\subsection{Assigning Values to Variables}\label{assigning-values-to-variables}

3 ways to do this:

\begin{description}
\item[Direct assignment] (like we just saw i.e. college=dkit)
\item[User Input] Prompt similar to \texttt{input}
\item[Positional Parameters] (arguments)
\end{description}


\section{User Input}\label{user-input}

\begin{itemize}

\item
  Shell scripts may require input from a user.
\item This input can be stored in a variable to be used later.
\item The \texttt{read} command is used to take input from standard input and place it in a variable.
\end{itemize}

\inputminted{bash}{read_example.sh}


\section{Positional Parameters (Arguments)}\label{positional-parameters-arguments}

\begin{itemize}

\item
  When we want to \emph{specify values on the command line} , we use
  \textbf{positional parameters or arguments} .
\item
  Positional Parameters are a series of \emph{special variables
  (\$0\ldots{}\$9) } that contain the contents of the values specified
  on the command line.
\item
  So for example if you want to read in the first name and second name
  of a person, \emph{you would pass the values while running the
  script.}
\item
  So if we write a script to echo greetings to a user and we want to
  pass the values for the user at run time, we replace the variable
  value with \texttt{\$1, \$2,} etc i.e.
\item
  \emph{echo ``Welcome'' \$1 \$2}
\item
  Then run the script and pass the values for \$1 and \$2 when running
  the script.
\item
  \emph{./script.sh } \emph{liz} \_ \_ \emph{frances} \_ \_
\end{itemize}

\section{Echo}\label{echo}

The echo command is used to add descriptions and spaces to script output:

\begin{minted}{bash}
college=dkit
echo "the college name is $college
\end{minted}

\end{document}