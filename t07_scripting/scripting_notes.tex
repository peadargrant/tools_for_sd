\documentclass[slides]{pgnotes}

\title{Scripting}

\begin{document}

\maketitle

\tableofcontents

\section{What is Scripting?}\label{what-is-scripting}

In CLI environments commands are used to carry out tasks (get-process, ls, mkdir, etc):
\begin{itemize}
\item These commands may be executed individually
\item May be combined using pipelining and redirection.
\end{itemize}
\begin{center}
  \textbf{Scripting adds familiar programming constructs to the command-line environment.}
\end{center}
\begin{itemize}
\item May need  to execute a number of commands together as a single unit.
\item Commands can be saved to a file - called a script.
\item Used to automate repetitive tasks.
\end{itemize}
  



\section{Scripting concepts}

\begin{itemize}
\item
  A script contains a number of commands that will be executed in the  order they are presented in.
\item
  Scripts are interpreted, line by line, as the script is run by the interpreter.
\item
  Error in scripts may result in scripts failing or making unexpected changes.
\end{itemize}

\section{Shells}

\begin{itemize}
\item
  Particular installation may have a number of different shells:
  \begin{description}
  \item[Windows:] CMD, PowerShell
  \item[Linux:] Bash, zsh, C-shell, ksh
  \item[Mac:] zsh, Bash
  \end{description}
\item
  Use cat /etc/shells to view the available shells    
\item
  Scripts are written for a particular shell (BASH, c shell, etc)
  \begin{itemize}
  \item
    First line of a script should identify the shell the script runs  on.
  \end{itemize}
\end{itemize}

\section{Bash}

\begin{itemize}
\item
  The default shell for Ubuntu is BASH.
\item
  Scripts should be saved with a .sh extension
\item
  Some text editors in Linux recognise scripts and format the text  accordingly.
\end{itemize}

\section{Facilities}\label{script-contents}

Scripts contain all the usual programming components:

\begin{itemize}
\item
  Variables
\item
  Input (from keyboard and via arguments)
\item
  Output (using echo)
\item
  Control Flow (looping -- for, while, case, etc)
\item
  Conditional Statements (if, ifelse, etc)
\item
  Functions
\end{itemize}

\section{Structure of a Script}\label{structure-of-a-script}

A script should follow the basic structure:

\inputminted{bash}{structure.sh}

\section{File Permissions}\label{file-permissions}

Knowledge of File Permission is essential for running scripts:

\begin{itemize}
\item
  By default, files cannot be executed.
\item
  Not even root can execute a file unless the file owner adds execute permission manually.
  \begin{itemize}
  \item
    This is a security feature.
  \end{itemize}
\item
  Use chmod and symbolic or numeric notation to change permissions:
  \begin{itemize}
  \item
    \texttt{chmod a+x filename}
  \item
    \texttt{chmod 777 filename}
  \end{itemize}
\end{itemize}

\section{Creating a Shell Script}\label{creating-a-shell-script}

\begin{enumerate}
\item Create a file using an editor
\item Start with header to specify interpreter:
  \begin{itemize}
  \item Normally \mintinline{latex}{#!/bin/bash}
  \end{itemize}
\item Good idea to have comment lines with purpose and your name
\item Then commands (and scripting constructs) follow.
\item Save the file with the .sh extension
\item Change permissions on file to execute it e.g. \mintinline{bash}{chmod +x ./myscript.sh}.
\item Run the file e.g. \mintinline{bash}{./myscript.sh} 
\end{enumerate}


\end{document}

