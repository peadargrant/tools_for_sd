\documentclass[slides]{pgnotes}

\title{cron scheduler}

\begin{document}

\maketitle

\tableofcontents

\section{Unattended execution}

There are many scenarios where a program needs to run unattended:

\begin{enumerate}
  
\item A Python script should be run every hour.
  It retrieves weather information and stores it in a database management system.

\item A bash script e-mails a report at 5PM each day using the \texttt{msmtp} program.

\item A Git repository needs to be \texttt{git pull}ed once a day at 6PM. 
  
\item A backup routine should be run every evening at 9PM.
  
\item A rogue employee wishes to send a mail at 6PM this evening with their resignation and post a tweet disparaging their employer!
  
\end{enumerate}


\subsection{Requirements}

We require that our command can run:
\begin{enumerate}
\item Without direct intervention at the shell
\item Without any user even logged in at all
\item Also we require that the program(s) will be run as required even if the system is restarted.   
\end{enumerate}

The command in question might be: 
\begin{enumerate}
\item A shell script (in Bash or another language)
\item A program installed from the package manager
\item A program we've written ourselves
\item Any combination thereof that we can write at the shell prompt!
\end{enumerate}

\subsection{Program requirements}

If a command is going to be run unattended, we need to make sure that:
\begin{enumerate}
\item Employs only \textbf{standard text-based} input / output
  \begin{itemize}
  \item Rules out ``full screen'' terminal-based programs. 
  \end{itemize}
\item Runs from start to finish in a relatively \textbf{linear path}
\item It \textbf{does not prompt} for or require interactive user input.
  Can sometimes use command-arguments to:
  \begin{itemize}
  \item Supply information that's normally prompted for
  \item Bypass confirmation questions 
  \end{itemize}
\item It doesn't hang (get stuck) in the event of errors.
  \begin{itemize}
  \item Can cause stuck runs to ``pile up'' consuming memory. 
  \end{itemize}
\end{enumerate}

\subsection{Options}

Scheduled execution generally falls into two distinct but similar scenarios:

\begin{description}
\item[Periodic] execution according to a repeating rule.
  \begin{itemize}
  \item Use the \texttt{cron} scheduler
  \end{itemize}
\item[One shot] execution at some point in the future.
  \begin{itemize}
  \item Use the \texttt{at} scheduler
  \item May also employ \texttt{cron} if a rule can be written.
  \end{itemize}
\end{description}

\section{cron scheduler}

The \textbf{cron} scheduler is a standard part of Linux and UNIX.
\begin{itemize}
\item It is controlled by means of the \texttt{crontab} file.
\item Can schedule execution based on:
  \begin{itemize}
  \item Date(s)
  \item Day of week
  \item Time (to 1 minute precision)
  \end{itemize}
\end{itemize}

\subsection{cron job}

A scheduled task is colloquially referred to as a ``cron job''.\\
We need to know:
\begin{enumerate}
\item The command to run
\item The working directory required
\item When the command is to be scheduled for execution at.
\end{enumerate}
Cron jobs normally execute as the user who creates them.
\begin{itemize}
\item Some cron jobs will need to run as root.
\end{itemize}


\section{at scheduler}

\end{document}

