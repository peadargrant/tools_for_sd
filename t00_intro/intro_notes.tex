\documentclass[slides]{pgnotes}

\title{Tools for software development\\Module introduction}

\begin{document}

\maketitle

\tableofcontents

\section{Aims}

\begin{itemize}
  \item This module introduces students to modern development tools, environments and processes.
  \item The industry is evolving rapidly and so are the tools.
  \item Understanding how to use these tools requires more knowledge and skills than just development, mathematics or theory.
\end{itemize}

\subsection{Changing environment}

Some of the drivers for these changes are:

\begin{itemize}
\item Complex systems with large data requirements are increasingly run on multiple computers so they can complete tasks quickly.
\item Companies are increasingly responsible for running software platforms for their clients and so have become far more focused on operating quality.
\item In order to have many teams work on larger systems, many companies are moving to platforms that distribute system functionality across different computers. 
\end{itemize}

\section{Challenges of Data Science Computing}

\subsection{Complexity}

\begin{itemize}
\item Data Science Applications can be complex and required to process very large quantities of data.
\item Algorithms can be mathematical and processor intensive.
\item Some datasets we wish to analyze can be very large and require a lot of resources (disk, memory, processing).
\item In modern environments we address these challenges by creating environments where we can run code on multiple computers at the same time (in parallel).
\end{itemize}

\subsection{Consistent environment}

\begin{itemize}
\item Complex programs uses many libraries and interacts with 3rd party software (like databases).
\item A challenge of all areas of computing is ensuring that environments are consistent.
\item It is difficult to ensure the environment where software is developed and tested is the same as the environment it runs in production:
  \begin{itemize}
  \item Libraries can be different versions
  \item 3rd party software can be different versions
  \item Operating system can be different versions
  \item Different hardware environments
  \end{itemize}
\item \textbf{``It works on my laptop!'' is no excuse!}
\end{itemize}

\subsection{Working in teams}

How do does a team:
\begin{enumerate}
\item Edit the same files while tracking changes
\item Show ownership / responsibility for code
\item Collaborate in on-site, fully remote and hybrid work environments
\item Automate routine tasks (like flagging code updates in MS Teams / Slack)
\item Centrally build, test and deploy code to production environment  
\end{enumerate}

\section{Key themes}

\subsection{Source control}

\begin{itemize}
\item Need to track revisions to files (primarily text-based).
\item Need ability to snapshot changes, or roll back.
\item Maintain a log of changes.
\item Show changes with each version.
\item Distributed source control:
  \begin{itemize}
  \item Developer(s) want to work on different systems.
  \item Permit collaboration by sharing changes amongst developers in a team.
  \end{itemize}
\item \textbf{We will use git in conjunction with GitLab for source control.}
\end{itemize}

\subsection{Automation}

\begin{itemize}
\item Your own \textbf{time is valuable}!
\item Don't waste it on tasks, processes that take longer than they should. 
\item Routine tasks should be automated to save time, improve consistency.
\item Familiarity with command-line environments required:
  \begin{itemize}
  \item Windows --- PowerShell
  \item Mac / Linux --- Bash, zsh
  \item Cross platform --- Python (re-use your programming knowledge!)
  \end{itemize}
  \textit{They may appear scary but are likely to be your most useful tool!}
\item \textbf{We will use some PowerShell and Bash to automate tasks.}
\end{itemize}


\subsection{Continuous integration}

\begin{enumerate}
\item Over the years the speed of releasing software has become more important (referred to as Velocity).
\item Businesses need software faster so they can be more competitive and not have business strategy held back by IT.
\item Businesses have also become far more dependent on software to run day to day business (Digital Transformation).  Any outages can have serious impact to business operation.
\item \textbf{Continuous integration:}
  \begin{itemize}
  \item automatically build, test and deploy code
  \item automatically run analysis, generating result artifacts
  \end{itemize}
\end{enumerate}


\subsection{Containers}

\begin{itemize}
\item By building all dependencies into a container, they can be run anywhere and will perform very consistently.
\item Multiple containers can be put onto their own private network so they can talk to each other.
\item This allows for micro service based systems to be built.
\item The most common container technology is called Docker.
\item \textbf{Docker} predominantly uses Linux Operating System.
\end{itemize}

\section{Assessment}

\vspace{\fill}

\begin{table}[hb]
  \caption{Assessment breakdown}
  \label{tab:assessment-breakdown}
  \begin{tabularx}{1.0\linewidth}{X r}
    \toprule
    \textbf{Component} & \textbf{Marks} \\
    \midrule
    Weekly lab exercises & 40 \\
    End-of-module project & 40 \\
    Class test & 20 \\
    \midrule
    \textbf{Total CA} & 100 \\
    \bottomrule
  \end{tabularx}
\end{table}

\vspace{\fill}


\subsection{Weekly lab work}

\begin{itemize}

\item
  You will be required to do some tasks each week during class in the lab.

\item
  You will use a source-control system (\texttt{git}) to keep these in a repository (on GitLab) which the lecturer will have access to.
  
\item
  You will be graded for completing the lab tasks throughout the semester. 

\item
  You can (and should!) help each other with lab work.

\item 
  \textbf{Your submitted work must be entirely your own}
  
\item
  If you miss a week, you should catch up as best you can.

\item
  Follow repository, folder naming and formatting requirements.\\
  If you don't you won't get any marks... Will be strict on this!
  
\end{itemize}

\end{document}

