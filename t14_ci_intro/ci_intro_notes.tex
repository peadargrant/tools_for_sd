\documentclass[slides]{pgnotes}

\title{Continuous integration}

\begin{document}

\maketitle

\tableofcontents

\section{Continuous integration}

Continuous integration technically refers to merging changes back into the main branch as often as possible:

\begin{itemize}
\item Idea originated to combine work of developers in teams.
\item Key idea (even for a solo developer) is that code is compiled, tested, packaged and deployed automatically on every commit.
\end{itemize}

Practical usage depends on:
\begin{description}
\item[Source control] system to record changes and trigger actions.
\item[CI toolset] to automate the process
\end{description}

\subsection{CI in software development}

CI can be used to automate any task, typically:

\begin{itemize}
\item Confirming that code compiles correctly without errors (for compiled languages)
\item Running automated tests.
\item Packaging code or executables into ZIP or other distribution formats (e.g. apk, pkg, MSI installers).
\item Generating other artefacts like screenshots, documentation, web pages etc.
\item Installing code on test (or production!) environments.
\item Announcing updated status using email, Slack, Teams, IRC, Twitter.
\item Updating metrics, dashboards, team visualisations etc. 
\end{itemize}


\subsection{CI for data science}

Some specific ideas for data science:

\begin{enumerate}
\item All of the previous ideas from software!
\item Converting markdown to different formats.
\item Generating visualisations using Python from input data.
\item Compiling a LaTeX document to PDF.
\item Data science lecturer creates course plan from topic folders.
\end{enumerate}


\subsection{Basic idea}

Regardless of how we perform our CI, the basic steps remain the same.
The CI tool should:

\begin{enumerate}
\item Cleanly clone the repository in full.
\item Run the steps specified.
\item Report the outcome of the steps.
\item Save / upload artefacts created.
\end{enumerate}

Reporting and saving can be done either by the CI tool itself or scripts in the repository.


\subsection{Requirements}

For any CI to work:

\begin{center}
\textbf{We must be able to run the build step as a sequence of deterministic commands, ideally a single command.}
\end{center}

How this works does not matter, but typically:

\begin{enumerate}
\item Makefile based on targets / dependencies using \texttt{make}.
\item Script in bash, PowerShell, Python, other language.
\end{enumerate}

Unless you can repeatably run your build steps without intervention, you won't be able to run them successfully under CI.


\subsection{Artefacts}

Normally we want to save some (not all) files generated during the build process.

\begin{itemize}
\item Reporting data about testing, performance, sample output etc.
\item Files intended for distribution:
  \begin{itemize}
  \item HTML or PDF from a markdown file
  \item Image output from a Python program on a CSV.
  \item MSI installer from source code for Windows
  \item APK app for an android mobile phone
  \end{itemize}
\end{itemize}

\subsection{CI tooling}

There are 3 options:

\begin{enumerate}

\item \textbf{Roll your own} using git, git hooks and bash / python scripts.

\item \textbf{Standalone CI server} like Jenkins, Travis.

\item \textbf{Integrated CI solutions} using source control hosts like GitLab, GitHub etc. 

\end{enumerate}

Git mainly used with GitLab / GitHub, so integrated option now makes sense in most cases. 


\section{GitLab CI}

To automate actions on commit we need:

\begin{description}
\item[Pipeline] that defines the steps to perform
\item[Executor] environment that will run the steps
\item[Runner] service to orchestrate:
  \begin{enumerate}
  \item Creating the executor
  \item Running the pipeline
  \end{enumerate}
\end{description}

\autoimage{ci_environment}{CI environment}{ci-environment}


\subsection{Pipeline}

The pipeline defines the commands to execute:
\begin{itemize}
\item Stored as file in repository as \texttt{.gitlab-ci.yml}
\item Pipelines vary in complexity:
  \begin{itemize}
  \item Simplest pipelines (as we'll use today) just run single (set of) commands when repository updated.
  \item More complex pipelines introduce depedencies, stages, rules etc.
  \end{itemize}
\end{itemize}



\subsection{Build machine}

Runner and executor are separate to GitLab itself:
\begin{itemize}
\item Need to be installed on a suitable computer (here a linux virtual machine) that
\item Should be on all the time, so can continuously communicate with GitLab.
\item Could be a machine on the network, in the cloud etc.
\end{itemize}

We will use our XOA Linux instance(s) to host the runner and provide the executor.


\subsection{Executor}

The executor is the environment used by the runner to run the pipeline.
Common executors:
\begin{description}
\item[Shell] executor uses the default system shell (e.g. bash on linux, PowerShell on Windows)
  \begin{itemize}
  \item Installed already so no further configuration needed.
  \item Issue: required packages etc need to be installed.
  \end{itemize}
\item[Docker] executor uses the docker container system (later)
  \begin{itemize}
  \item Needs to be separately setup.
  \item Advantage: build environment can be setup in repository.
  \end{itemize}
\end{description}
For today we'll use the shell executor.


\subsection{Runner}

The runner needs to be installed on our XOA instance:
\begin{itemize}
\item The runner communicates with GitLab continuously.
\item When new commits are pushed to the repository the runner:
  \begin{itemize}
  \item Invokes the pipeline using the executor
  \item Reports progress back to GitLab for display
  \item Captures defined artefacts and sends them to GitLab
  \end{itemize}
\end{itemize}
  
\end{description}



\end{document}

 
